%%%%%%%%%%%%%%%%%%%%%%%%%%%%%%%%%%%%%%%%%%%%%%%%%
\documentclass[12pt,notitlepage]{article}
\usepackage{cite}
\usepackage{amsmath,amssymb}
\usepackage{comment}
\usepackage{multirow}
\usepackage[utf8]{inputenc}
\usepackage{bm}
\usepackage{tgheros} % Use Helvetica
\usepackage{fancyhdr}
\usepackage[dvipdfmx]{graphicx, hyperref, xcolor}

%%%%% Hyperref %%%%%
\definecolor{orangered}{HTML}{FF4500}
\definecolor{crimson}{HTML}{DC143C}
\definecolor{rossoferrari}{HTML}{D9073D}
\definecolor{steelblue}{HTML}{4682B4}
\definecolor{mediumblue}{HTML}{0000CD}
\definecolor{forestgreen}{HTML}{228B22}
\hypersetup{% hyperref option list
setpagesize=false,
bookmarksnumbered=true,%
bookmarksopen=true,%
colorlinks=true,%
linkcolor=orangered,
urlcolor=steelblue,
citecolor=steelblue,
}

%%%%% spacing %%%%%
\renewcommand{\baselinestretch}{1.48}

%%%%% Geometry %%%%%
\usepackage[height=21.5cm,width=16.5cm,centering]{geometry}

%%%%% New commands %%%%%
\newcommand{\Slash}[1]{{\ooalign{\hfil/\hfil\crcr$#1$}}}
\newcommand{\hyphen}{\,\mathchar`-\mathchar`-\,}

\renewcommand{\thepage}{\arabic{page}}
\setcounter{page}{1}
%\renewcommand{\thefootnote}{\#\arabic{footnote}}
\renewcommand{\thefootnote}{$\natural$\arabic{footnote}}
\setcounter{footnote}{0}

%%%%% header %%%%%
\pagestyle{fancy}
\rhead[Research Statement $-$ So Chigusa]{Research Statement $-$ So Chigusa}
\renewcommand{\headrulewidth}{0pt} % vanishing hr
\renewcommand{\footrulewidth}{0pt}

%%%%%%%%%%%%%%%%%%%%%%%%%%%%%%%
%%%    remove the following commands when finalizing
%%%%%%%%%%%%%%%%%%%%%%%%%%%%%%%
\def\rem#1{ {\bf\textcolor{red}{($\clubsuit$ #1 $\clubsuit$)}}}
%%%%%%%%%%%%%%%%%%%%%%%%%%%%%%%
%%%%%%%%%%%%%%%%%%%%%%%%%%%%%%%

% \allowdisplaybreaks[4]

\title{\vspace*{-3cm}Research Statement}
\author{\textbf{So Chigusa}}
\date{\vspace*{-4mm}\textit{Ph.D. candidate in Physics at University of Tokyo}}

\begin{document}
\maketitle

My research interests lie in the phenomenology of a broad range of models beyond the standard model (SM).
In spite of the great success of the SM, there remain many problems that cannot be solved within the SM.
These problems include, for example, the existence of the dark matter (DM), the naturalness of the electroweak (EW) scale, the structure of the gauge symmetries, and the origin of the quark and lepton families.
To answer one or several of them, many interesting models are proposed such as the minimally supersymmetric standard model (MSSM), the grand unified theory (GUT), flavor symmetry, and so on.

In my opinion, many of these models are still unsatisfactory, since they sometimes have inconsistencies with experimental results or astronomical and cosmological observations, or have too many degrees of freedom to obtain a non-trivial prediction.
Thus, in my research career, I seek ways to test the models and, at the same time, try to construct new models or mechanisms that solve the above problems simply and reasonably.
My main achievements described in the following are categorized as
\vspace{-1.3mm}
\begin{itemize}
  \setlength{\parskip}{0mm}
  \setlength{\itemsep}{1mm}
  \item precise calculation of the EW vacuum decay rate to test the SM and beyond.
  \item analysis of the search for the MSSM particles using future hadron colliders.
  \item analysis of the astrophysical constraint on the axion-like DM.
  \item construction of cosmologically viable models with discrete flavor symmetry.
  \item search for the parameter space of the MSSM compatible with the GUT prediction.
\end{itemize}

\vspace*{-2mm}
\subsection*{Achievements so far}

During the last few years, I have been developing the next-to-leading order calculation of the decay rate of the EW vacuum.
This approach is powerful because it applies to many models that contain some additional fields and/or couplings to the SM Higgs boson even when new particles are too heavy to be accessed using any ongoing or planned experiment.
In this topics, I made several achievements:
\vspace{-1.3mm}
\begin{itemize}
  \setlength{\parskip}{0mm}
  \setlength{\itemsep}{1mm}
  \item
    I considered the SM and gave a proper way to treat the flat direction of the Euclidean action related to the classical conformal invariance.
    This enabled us to precisely evaluate the decay rate with error estimations. ([11] of my publications)
  \item
    I considered models with new particles that couple to the SM Higgs and obtained severe constraints on their couplings and masses from the EW vacuum stability. [10]
  \item
    I proposed a new way based on the gradient flow to numerically evaluate the multi-field bounce configuration.
    This method is useful for the next-leading order calculation of the vacuum decay rate when the asymptotic behavior of the bounce is needed. [3]
\end{itemize}

It is also important to use the collider experiments to search for the new physics.
In particular, hadron colliders provide a huge amount of data, under which a hint of the new physics may be buried.
To fully use the data, it is necessary to develop a proper physics quantity or a wise way to extract the signal of a new particle.
I have adopted several ways to tackle this difficult problem using future hadron colliders.
\vspace{-1.3mm}
\begin{itemize}
  \setlength{\parskip}{0mm}
  \setlength{\itemsep}{1mm}
  \item
    I focused on the parameter space of the MSSM where wino is the lightest new physics particle.
    I used the so-called disappearing track signal caused by sufficiently long-lived charged winos to reduce background events.
    In this work, I collaborated with several experimentalists and sought a possibility of using track information to fully reconstruct the event kinematics and to determine the masses of new particles [6] and the wino lifetime [1].
  \item
    When we search for the Higgsino-like state in the MSSM, the charged component is short-lived and the track information is difficult to use.
    To handle this case, I focused on the lepton pair production process that is affected by Higgsino at the one-loop level and developed a way to use the energy dependence of the cross section to reduce the systematic uncertainties.
    As a result, I obtained the best limit so far for Higgsino [5,7].
\end{itemize}

The existence of the DM is also a guideline for the direction of new physics search and there have been many interesting DM models in the literature.
In spite of many efforts made to look for these DM candidates, a wide region of parameter space, in particular, the relatively light mass region, remains unexplored.
I studied a way to cover some of such an unexplored region using astronomical observations.
\vspace{-1.3mm}
\begin{itemize}
  \setlength{\parskip}{0mm}
  \setlength{\itemsep}{1mm}
  \item
    I focused on the axion-like particle (ALP) DM with a very light mass, which may be ubiquitous in string theory and motivated by the structure of the DM halo.
    I used the rotation of the polarization plane of the linearly polarized light caused by the existence of ALPs, in particular, its time-dependence due to the coherent oscillation of ALPs, to constrain the parameter space.
    I obtained, depending on the light source and the ALP mass, a prospect severer than existing bounds. [2]
\end{itemize}

It is another interesting way of research to build a new physics model.
All problems in the SM are hints of the theory of everything, which is our ultimate goal.
By providing simple solutions to them, we can enrich our knowledge about what could be there beyond the SM.
So far, I worked on discrete flavor symmetry and GUT models that provide a unified description of SM fermion families and gauge symmetries, respectively.
\vspace{-1.3mm}
\begin{itemize}
  \setlength{\parskip}{0mm}
  \setlength{\itemsep}{1mm}
  \item
    In models with spontaneously broken discrete symmetries, domain walls may be formed that will eventually dominate the energy of our universe.
    To avoid this problem, I constructed several models with discrete flavor symmetry in which cosmological dynamics do not lead to the formation of the stable domain wall.
    In my models, the most recent observation of the neutrino mixing angles is correctly predicted. [4,8,9]
  \item
    I focused on the supersymmetric $\mathrm{SU}(5)$ GUT and looked for the parameter space where the unification of the Yukawa coupling constants, which is the prediction of the model, is achieved.
    I also considered the effect of extra matter fields by calculating their contribution to the renormalization group equations and found some parameter space that gives a consistent prediction. [12,13]
\end{itemize}

\vspace*{-2mm}
\subsection*{Future plans}

Now it is an exciting time with many ongoing and future planned experiments that provide a huge number of hints of the new physics.
The results of these experiments will guide future theoretical works.
At the same time, a great deal of effort is devoted to developing new approaches to the problems of the SM and new techniques to extract the information from experimental results.
As a young researcher at this time, I am eager to engage in fields of great progress and to keep seeking a trace of the new physics.

In the near future, I will extend my calculation of the EW vacuum decay rate to more complicated models with several scalar fields involved in the bounce.
The extension will allow us to apply our calculation to numerous models.
I will also consider how to take account of the thermal effect and effect of gravity both of which may drastically change the calculation.

Besides, I will analyze the pair production process of gauge bosons to search for Higgsino.
This analysis combined with my previous ones increases statistics, which may allow us to reach the Higgsino mass preferred from the DM relic abundance.
Another possibility is to study loop topologies different from the vacuum polarization.
Through the classification of the energy dependence of the cross section for each topology, I will look for different particles to which our method can be applied.

Recently, a lot of techniques are developed that enable us to systematically analyze the physics models and experimental data.
These include, for example, the effective field theory (EFT) and machine learning.
I am also interested in learning these techniques and considering their applications to the new physics search.
One possible application will be the collider phenomenology, in particular, the understanding of the behavior of jets and the identification of the observed jets.
Another interesting application will be the calculation of the event rate in DM direct detection experiments, where the EFT technique will help us to treat many models with a broad range of DM mass systematically.

After all, the fields of great interest highly depend on the future development of the calculation technique, appearance of new ideas, and results of experiments.
To keep track of the latest trends, I will not restrict myself to studying the same subjects as I have been involved in and have a wide field of view during my research career.

\end{document}
