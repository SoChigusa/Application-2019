%%%%%%%%%%%%%%%%%%%%%%%%%%%%%%%%%%%%%%%%%%%%%%%%%
\documentclass[12pt,notitlepage]{book}
% \pdfoutput=1

\usepackage{cite}
\usepackage{amsmath,amssymb}
\usepackage{comment}
\usepackage{multirow}
\usepackage[utf8]{inputenc}
\usepackage{bm}
\usepackage{accents}
\usepackage{fancyhdr}

%%%%% Graphics %%%%%
\usepackage{ifpdf}
\ifpdf
  \usepackage{graphicx, hyperref, xcolor}     %   usepackage without driver option
\else     % For (p)LaTeX + dvipdfmx
  \usepackage[dvipdfmx]{graphicx, hyperref, xcolor}     %   usepackage with driver option
 \fi
\usepackage{subcaption}

%%%%% Hyperref %%%%%
\definecolor{orangered}{HTML}{FF4500}
\definecolor{crimson}{HTML}{DC143C}
\definecolor{rossoferrari}{HTML}{D9073D}
\definecolor{steelblue}{HTML}{4682B4}
\definecolor{mediumblue}{HTML}{0000CD}
\definecolor{forestgreen}{HTML}{228B22}
\hypersetup{% hyperref option list
setpagesize=false,
bookmarksnumbered=true,%
bookmarksopen=true,%
colorlinks=true,%
linkcolor=orangered,
urlcolor=steelblue,
citecolor=steelblue,
}

%%%%% spacing %%%%%
\renewcommand{\baselinestretch}{1.5}

%%%%% Geometry %%%%%
\usepackage[height=21.5cm,width=16.5cm,centering]{geometry}

%%%%% New commands %%%%%
\newcommand{\Slash}[1]{{\ooalign{\hfil/\hfil\crcr$#1$}}}
\newcommand{\hyphen}{\,\mathchar`-\mathchar`-\,}

\renewcommand{\thepage}{\arabic{page}}
\setcounter{page}{1}
%\renewcommand{\thefootnote}{\#\arabic{footnote}}
\renewcommand{\thefootnote}{$\natural$\arabic{footnote}}
\setcounter{footnote}{0}

%%%%% header %%%%%
\pagestyle{fancy}
\rhead[Research statement $-$ So Chigusa]{}
\renewcommand{\headrulewidth}{0pt} % vanishing hr

%%%%%%%%%%%%%%%%%%%%%%%%%%%%%%%
%%%    remove the following commands when finalizing
%%%%%%%%%%%%%%%%%%%%%%%%%%%%%%%
\def\rem#1{ {\bf\textcolor{red}{($\clubsuit$ #1 $\clubsuit$)}}}
%%%%%%%%%%%%%%%%%%%%%%%%%%%%%%%
%%%%%%%%%%%%%%%%%%%%%%%%%%%%%%%

\allowdisplaybreaks[1]

\title{\vspace*{-3cm}Research statement}
\author{\textbf{So Chigusa}}
\date{\vspace*{-4mm}\textit{Ph.D. candidate in Physics at University of Tokyo}}

\begin{document}
\maketitle

My current research concentrates on the search for new particles with
the electroweak (EW) interaction, the so-called EWIMPs, with the
$\mathrm{TeV}$-scale mass.  My main achievements are evaluating their
effects on the EW vacuum stability and collider phenomenology.  I am
planning to extend my calculation so that I can analyze a broader range
of models.

\vspace*{-2mm}
\subsection*{Achievements so far}

There are several theoretical motivations to consider
$\mathrm{TeV}$-scale EWIMPs.  They appear in various new physics models
that explain the energy scale of the EW symmetry breaking.  Besides,
they can naturally explain the relic abundance of the dark matter (DM).
Well-motivated examples of EWIMPs are Higgsino and Wino in the minimally
supersymmetric standard model (MSSM), which are the superpartners of the
standard model (SM) Higgs and W boson, respectively.  Since their
existence significantly modifies the fate of the EW vacuum and the
phenomenology at the high energy scale, considerable efforts have been
devoted, but the results are still unsatisfactory.

During the last few years, we have been developing the next-to-leading
order calculation of the decay rate of the EW vacuum.  We apply a
recently proposed calculation which is manifestly gauge-invariant to the
models that contain the SM-like Higgs as a unique scalar field.  We
assume a purely quartic Higgs potential and provide the correct
treatment of the flat direction of the Euclidean action related to the
classical conformal invariance of the potential.  This treatment fills a
gap of the existing calculation and enables us to precisely evaluate the
decay rate with error estimations.  We analyze not only the SM but also
several models with new particles that couple to Higgs and obtain severe
constraints on the coupling and mass of new particles even for the mass
region much heavier than the $\mathrm{TeV}$-scale.

Recently, we have also worked on the new physics search and measurement
using future hadron colliders.  Although EWIMPs possess charged
components that interact with detector materials, these components may
be too short-lived to detect, which is sometimes the case for Higgsino.
To probe such short lifetime EWIMPs, we consider the vacuum polarization
effect from the EWIMP loop and its effects on the lepton pair production
process.  The energy dependence of the loop effect possesses a
characteristic dip structure around the EWIMP pair production threshold.
We use this shape to distinguish the signal from the background and
effects of systematic errors and obtain the best limit so far for the
short lifetime Higgsino.  We also reveal that the signal shape can be
used for measurement of the coupling and mass of discovered EWIMPs.

% We have adopted several ways to suppress the systematic errors and
% background events arising from the strong interaction between quarks.
% One is to use the disappearing charged track signal, focusing on the
% models with long-lived charged particles.  In this project, I
% collaborate with several experimentalists and apply this method to the
% supersymmetric extension of the SM with anomaly mediated supersymmetry
% breaking.  By combining the charged track timing information and the
% transverse momentum conservation, we fully reconstruct the kinematics of
% the new physics events and estimate the errors in mass measurement of
% the new physics particles.

% \subsection*{Ongoing projects}

% Bounce from gradient flow??

% \rem{Long version: citation, author names}

\vspace*{-2mm}
\subsection*{Future plans}

In the near future I will extend our calculation of the vacuum decay
rate to the models with several scalar fields involved in the bounce
configuration.  This allows us to evaluate the vacuum decay rate in
various complicated models such as the MSSM, and to constrain the
parameter space even when the relevant scalar particles are beyond the
experimental reach.  For the generic models with several mass scales,
the multi-scale nature of the bounce makes the numerical calculation
difficult.  Therefore, another possible future direction is to develop
an algorithm to adaptively adjust the lattice spacing for the fast and
accurate computation.

Regarding the collider phenomenology, I will apply our method to the
pair production process of gauge bosons.  This gives us a severer bound
on the Higgsino that may reach the mass prefered from the DM relic
abundance.  Another possibility is to study loop topologies different
from the vacuum polarization.  Through the classification of the signal
shape for each loop topology, it will become possible to find new
particles to which our detection method can be applied.  Finally, the
future planned lepton colliders such as ILC and CLIC also bring us some
exciting possibilities.  Since they provide more precise measurements in
lower energy scale compared with hadron colliders, it will be efficient
to analyze the new particles effect using the effective field theory
approach.  I will provide some model-independent constraints on EWIMPs
through the precise measurement at lepton colliders.

\end{document}
