%%%%%%%%%%%%%%%%%%%%%%%%%%%%%%%%%%%%%%%%%%%%%%%%%
\documentclass[12pt,notitlepage]{article}
\usepackage{cite}
\usepackage{amsmath,amssymb}
\usepackage{comment}
\usepackage{multirow}
\usepackage[utf8]{inputenc}
\usepackage{bm}
\usepackage{tgheros} % Use Helvetica
\usepackage{fancyhdr}
\usepackage[dvipdfmx]{graphicx, hyperref, xcolor}

%%%%% Hyperref %%%%%
\definecolor{orangered}{HTML}{FF4500}
\definecolor{crimson}{HTML}{DC143C}
\definecolor{rossoferrari}{HTML}{D9073D}
\definecolor{steelblue}{HTML}{4682B4}
\definecolor{mediumblue}{HTML}{0000CD}
\definecolor{forestgreen}{HTML}{228B22}
\hypersetup{% hyperref option list
setpagesize=false,
bookmarksnumbered=true,%
bookmarksopen=true,%
colorlinks=true,%
linkcolor=orangered,
urlcolor=steelblue,
citecolor=steelblue,
}

%%%%% spacing %%%%%
\renewcommand{\baselinestretch}{1.5}

%%%%% Geometry %%%%%
\usepackage[height=21.5cm,width=16.5cm,centering]{geometry}

%%%%% New commands %%%%%
\newcommand{\Slash}[1]{{\ooalign{\hfil/\hfil\crcr$#1$}}}
\newcommand{\hyphen}{\,\mathchar`-\mathchar`-\,}

\renewcommand{\thepage}{\arabic{page}}
\setcounter{page}{1}
%\renewcommand{\thefootnote}{\#\arabic{footnote}}
\renewcommand{\thefootnote}{$\natural$\arabic{footnote}}
\setcounter{footnote}{0}

%%%%% header %%%%%
\pagestyle{fancy}
\rhead[Research statement $-$ So Chigusa]{Research statement $-$ So Chigusa}
\renewcommand{\headrulewidth}{0pt} % vanishing hr
\renewcommand{\footrulewidth}{0pt}

%%%%%%%%%%%%%%%%%%%%%%%%%%%%%%%
%%%    remove the following commands when finalizing
%%%%%%%%%%%%%%%%%%%%%%%%%%%%%%%
\def\rem#1{ {\bf\textcolor{red}{($\clubsuit$ #1 $\clubsuit$)}}}
%%%%%%%%%%%%%%%%%%%%%%%%%%%%%%%
%%%%%%%%%%%%%%%%%%%%%%%%%%%%%%%

%\allowdisplaybreaks[1]

\title{\vspace*{-3cm}Research statement}
\author{\textbf{So Chigusa}}
\date{\vspace*{-4mm}\textit{Ph.D. candidate in Physics at University of Tokyo}}

\begin{document}
\maketitle

My research interests lie in the phenomenology of a broad range of models beyond the standard model (SM) of particle physics.
In spite of the great success of the SM, there remain many problems that cannot be solved within the SM.
These questions include, for example, the existence of the dark matter (DM), naturalness of the electroweak (EW) scale, structure of the gauge symmetries, and origin of the quark and lepton families.
To answer one or several of them, numbers of models are proposed such as the minimally supersymmetric standard model (MSSM), grand unified theory (GUT), flavor symmetry, and so on.
They often lead to some interesting phenomenology that can be used to distinguish them from the SM.
I seek ways to test them using both a top-down approach where a model or its parameter space is constrained from theoretical consideration, and a bottom-up approach where the experimental search probes a new particle contained in a model.

\vspace*{-2mm}
\subsection*{Achievements so far}

%There are several theoretical motivations to consider $\mathrm{TeV}$-scale EWIMPs.  They appear in various new physics models that explain the energy scale of the EW symmetry breaking.
%Besides, they can naturally explain the relic abundance of the dark matter (DM).
%Well-motivated examples of EWIMPs are Higgsino and Wino in the minimally supersymmetric standard model (MSSM), which are the superpartners of the standard model (SM) Higgs and W boson, respectively.
%Since their existence significantly modifies the fate of the EW vacuum and the phenomenology at the high energy scale, considerable efforts have been devoted, but the results are still unsatisfactory.

Since we now live in the EW vacuum, the requirement of the (meta-)stablity of the vacuum can be used to test and constrain models in a top-down approach.
This approach is powerful because it can probe new particles that are too heavy to be accessed using any ongoing or planned experiment.
Also, this approach is applicable to many models which contain some additional scalar fields and/or couplings to the SM Higgs boson.
During the last few years, I have been developing the next-to-leading order calculation of the decay rate of the EW vacuum ([1,8,9] of my publication list).
%I assumed that the SM Higgs is the unique scalar particle in the model and provided a correct treatment of the flat direction of the Euclidean action related to the approximate classical conformal invariance of the potential.
My treatment filled a gap of existing calculations and enabled us to precisely evaluate the decay rate with error estimations.
%I analyzed not only the SM but also models with new particles that couple to Higgs and obtained severe constraints on the couplings and masses of new particles even for the mass region that cannot be accessed using any ongoing or planned experiment.

It is also important to use the collider experiments as an example of the bottom-up approach.
In particular, recent hadron colliders provide a huge amount of data, under which a hint of the new physics may be buried.
To fully use the data, it is necesarry to develop a proper physics quantity or a wise way to extract the signal of a new particle.
So far, I focused on the search for massive particles with EW charges (EWIMPs) [3--5] that are DM candidates appearing in many well-known models such as MSSM.
%This is because $\mathrm{TeV}$-scale EWIMPs naturally explain the relic abundance of the DM if the non-thermal production can be neglected.
%Recently, I have worked on the EWIMP search and measurement of its properties using future hadron colliders ([3--5]).
%In particular, I focused on EWIMPs with short-lived charged components such as the Higgsino-like state in the MSSM.
%Since the tracker information cannot be used in this case, instead I considered the vacuum polarization effect from the EWIMP loop on the lepton pair production process.
%I revealed that the energy dependence of the loop effect possesses a characteristic shape with a sharp peak and used this shape to distinguish the signal from the background and effects of systematic errors.
%As a result, I obtained the best limit so far for the short lifetime Higgsino and revealed that the signal shape can also be used for measurement of the coupling and mass of discovered EWIMPs.
I developed a way to use the signal shape to reduce the systematic uncertainties and obtained the best limit so far for Higgsino, an EWIMP DM candidate in MSSM, which are generally difficult to search for.

% I have adopted several ways to suppress the systematic errors and background events arising from the strong interaction between quarks.
% One is to use the disappearing charged track signal, focusing on the models with long-lived charged particles.
% In this project, I collaborate with several experimentalists and apply this method to the supersymmetric extension of the SM with anomaly mediated supersymmetry breaking.
% By combining the charged track timing information and the transverse momentum conservation, I fully reconstruct the kinematics of the new physics events and estimate the errors in mass measurement of the new physics particles.

It is an another interesting way of research to search for or to build a new physics model.
All problems in the SM are hints of the most fundamental physics theory, which is our ultimate goal.
By providing a simple solution to them, we can enrich our knowleadge about what could be there beyond the SM.
In several works, I focused on models with discrete flavor symmetry [2,6,7] and GUT models [10,11] that provide an unified description for SM fermion families and gauge symmetries, respectively.
I constructed models and looked for the parameter space in which model predictions are consistent with our cosmological history and experimental results.

% \subsection*{Ongoing projects}

% Bounce from gradient flow??

% \rem{Long version: citation, author names}

\vspace*{-2mm}
\subsection*{Future plans}

%In the near future, I will extend our calculation of the vacuum decay rate to models with several scalar fields involved in the bounce configuration.
%This allows us to evaluate the vacuum decay rate in various complicated models such as the MSSM, and to constrain the parameter space even when the new particles are beyond the experimental reach.
%For the generic models with several mass scales, the multi-scale nature of the bounce makes the numerical calculation difficult.
%Therefore, another possible future direction is to develop an algorithm to adaptively adjust the lattice spacing for the fast and accurate computation.
%Regarding the collider search, I will apply our method to the pair production process of gauge bosons.
%This analysis results in the increase in statistics, which may allow us to reach the Higgsino mass preferred from the DM relic abundance.
%Another possibility is to study loop topologies different from the vacuum polarization.
%Through the classification of the signal shape for each topology, I will look for new particles to which our method can be applied.
%Finally, the future planned lepton colliders such as ILC and CLIC also bring us some exciting possibilities.
%Since they provide more precise measurements in lower energy scale compared with hadron colliders, it will be efficient to analyze the new particles effect using the effective field theory approach.
%I will provide some model-independent constraints on EWIMPs through the precise measurement at lepton colliders.

Now it is an exciting time with many ongoing and future planned experiments that provide a huge number of hints of the new physics.
The results of these experiments will guide future theoretical works.
At the same time, a great deal of effort is devoted to developing new approaches to the problems of the SM and new techniques to extract the information from experimental results.
As a young researcher at this time, I am eager to engage in fields of great progress and to keep seeking a trace of the new physics.

Recently, we have made significant progress in developing techniques that enable us to analyze the collider data in a systematic way.
These include, for example, the effective field theory and machine learning.
One possible way is to consider how to use them to enlarge the reach of the collider experiments.
Another possibility is

\end{document}
