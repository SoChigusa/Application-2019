%%%%%%%%%%%%%%%%%%%%%%%%%%%%%%%%%%%%%%%%%%%%%%%%%
\documentclass[12pt,notitlepage]{book}
% \pdfoutput=1

\usepackage{cite}
\usepackage{amsmath,amssymb}
\usepackage{comment}
\usepackage{multirow}
\usepackage[utf8]{inputenc}
\usepackage{bm}
\usepackage{accents}
\usepackage{fancyhdr}

%%%%% Graphics %%%%%
\usepackage{ifpdf}
\ifpdf
  \usepackage{graphicx, hyperref, xcolor}     %   usepackage without driver option
\else     % For (p)LaTeX + dvipdfmx
  \usepackage[dvipdfmx]{graphicx, hyperref, xcolor}     %   usepackage with driver option
 \fi
\usepackage{subcaption}

%%%%% Hyperref %%%%%
\definecolor{orangered}{HTML}{FF4500}
\definecolor{crimson}{HTML}{DC143C}
\definecolor{rossoferrari}{HTML}{D9073D}
\definecolor{steelblue}{HTML}{4682B4}
\definecolor{mediumblue}{HTML}{0000CD}
\definecolor{forestgreen}{HTML}{228B22}
\hypersetup{% hyperref option list
setpagesize=false,
bookmarksnumbered=true,%
bookmarksopen=true,%
colorlinks=true,%
linkcolor=orangered,
urlcolor=steelblue,
citecolor=steelblue,
}

%%%%% spacing %%%%%
\renewcommand{\baselinestretch}{1.5}

%%%%% Geometry %%%%%
\usepackage[height=21.5cm,width=16.5cm,centering]{geometry}

%%%%% New commands %%%%%
\newcommand{\Slash}[1]{{\ooalign{\hfil/\hfil\crcr$#1$}}}
\newcommand{\hyphen}{\,\mathchar`-\mathchar`-\,}

\renewcommand{\thepage}{\arabic{page}}
\setcounter{page}{1}
%\renewcommand{\thefootnote}{\#\arabic{footnote}}
\renewcommand{\thefootnote}{$\natural$\arabic{footnote}}
\setcounter{footnote}{0}

%%%%% header %%%%%
\pagestyle{fancy}
\rhead[Research statement $-$ So Chigusa]{}
\renewcommand{\headrulewidth}{0pt} % vanishing hr

%%%%%%%%%%%%%%%%%%%%%%%%%%%%%%%
%%%    remove the following commands when finalizing
%%%%%%%%%%%%%%%%%%%%%%%%%%%%%%%
\def\rem#1{ {\bf\textcolor{red}{($\clubsuit$ #1 $\clubsuit$)}}}
%%%%%%%%%%%%%%%%%%%%%%%%%%%%%%%
%%%%%%%%%%%%%%%%%%%%%%%%%%%%%%%

\allowdisplaybreaks[1]

\title{\vspace*{-3cm}Research statement}
\author{\textbf{So Chigusa}}
\date{\vspace*{-4mm}\textit{Ph.D. candidate in Physics at University of Tokyo}}

\begin{document}
\maketitle

My research interests lie in the phenomenology of broad range of models beyond the standard model (SM) of the particle physics.
In spite of the great success of the SM, there remain many questions that cannot be answered within the SM.
These questions include, for example, the structure of the gauge symmetries, the existence of the dark matter (DM), origin of the quark and lepton families, and naturalness of the electroweak (EW) scale.
To give an answer to one or several of them, numbers of models are proposed such as the grand unification, minimally supersymmetric extention of the SM (MSSM), family symmetry, and so on.
They often lead to some interesting phenomenology that can be used to distinguish them from the SM.
I seek ways to test them using both a top-down approach where a model or its parameter space is constrained from theoretical consideration, and a bottom-up approach where the experimental search probes a new particle contained in a model.

\vspace*{-2mm}
\subsection*{Achievements so far}

%There are several theoretical motivations to consider $\mathrm{TeV}$-scale EWIMPs.  They appear in various new physics models that explain the energy scale of the EW symmetry breaking.
%Besides, they can naturally explain the relic abundance of the dark matter (DM).
%Well-motivated examples of EWIMPs are Higgsino and Wino in the minimally supersymmetric standard model (MSSM), which are the superpartners of the standard model (SM) Higgs and W boson, respectively.
%Since their existence significantly modifies the fate of the EW vacuum and the phenomenology at the high energy scale, considerable efforts have been devoted, but the results are still unsatisfactory.

Many models contain some additional scalar fields and/or coupling to the SM Higgs boson that affect the stability of the EW vacuum.
Requirement of the (meta-)stable EW vacuum can be used to test and constrain models.
During the last few years, we have been developing the next-to-leading order calculation of the decay rate of the EW vacuum (please see [1,8,9] of my publication list).
We assumed that the SM Higgs is the unique scalar particle in the model and provided a correct treatment of the flat direction of the Euclidean action related to the approximate classical conformal invariance of the potential.
This treatment filled a gap of the existing calculation and enabled us to precisely evaluate the decay rate with error estimations.
We analyzed not only the SM but also several models with new particles that couple to Higgs and obtained severe constraints on the coupling and mass of new particles even for the mass
region much heavier than the $\mathrm{TeV}$-scale.

As for the collider search, massive particles with EW charges (EWIMPs) are interesting targets that often appear in models since the $\mathrm{TeV}$-scale mass is predicted from the relic abundance of the DM neglecting the non-thermal production.
Recently, we have worked on the EWIMP search and measurement of its properties using future hadron colliders (please see [3--5]).
In particular, we focused on EWIMPs with short lived charged components such as the Higgsino-like state in the MSSM.
Since the tracker information cannot be used in this case, instead we considered the vacuum polarization effect from the EWIMP loop and its effects on the lepton pair production process.
We revealed that the energy dependence of the loop effect possesses a characteristic shape with a sharp peak and used this shape to distinguish the signal from the background and effects of systematic errors.
As a result, we obtained the best limit so far for the short lifetime Higgsino and revealed that the signal shape can also be used for measurement of the coupling and mass of discovered EWIMPs.

% We have adopted several ways to suppress the systematic errors and
% background events arising from the strong interaction between quarks.
% One is to use the disappearing charged track signal, focusing on the
% models with long-lived charged particles.  In this project, I
% collaborate with several experimentalists and apply this method to the
% supersymmetric extension of the SM with anomaly mediated supersymmetry
% breaking.  By combining the charged track timing information and the
% transverse momentum conservation, we fully reconstruct the kinematics of
% the new physics events and estimate the errors in mass measurement of
% the new physics particles.

% \subsection*{Ongoing projects}

% Bounce from gradient flow??

% \rem{Long version: citation, author names}

\vspace*{-2mm}
\subsection*{Future plans}

In the near future I will extend our calculation of the vacuum decay
rate to the models with several scalar fields involved in the bounce
configuration.  This allows us to evaluate the vacuum decay rate in
various complicated models such as the MSSM, and to constrain the
parameter space even when the relevant scalar particles are beyond the
experimental reach.  For the generic models with several mass scales,
the multi-scale nature of the bounce makes the numerical calculation
difficult.  Therefore, another possible future direction is to develop
an algorithm to adaptively adjust the lattice spacing for the fast and
accurate computation.

Regarding the collider phenomenology, I will apply our method to the
pair production process of gauge bosons.  This gives us a severer bound
on the Higgsino that may reach the mass prefered from the DM relic
abundance.  Another possibility is to study loop topologies different
from the vacuum polarization.  Through the classification of the signal
shape for each loop topology, it will become possible to find new
particles to which our detection method can be applied.  Finally, the
future planned lepton colliders such as ILC and CLIC also bring us some
exciting possibilities.  Since they provide more precise measurements in
lower energy scale compared with hadron colliders, it will be efficient
to analyze the new particles effect using the effective field theory
approach.  I will provide some model-independent constraints on EWIMPs
through the precise measurement at lepton colliders.

\end{document}
